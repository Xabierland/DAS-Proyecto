\documentclass[a4paper,12pt]{report}
\usepackage[spanish]{babel}
\usepackage[utf8]{inputenc}
\usepackage{graphicx, csquotes, longtable, array, booktabs, xparse, float, titlesec, enumitem, dingbat, soul, multicol, listings}
\usepackage[dvipsnames]{xcolor}
\usepackage[margin=2cm]{geometry}

% Añadir la bibliografía
%\usepackage[backend=biber, style=numeric, sorting=ynt]{biblatex}
%\addbibresource{TFG.bib}

% Cambia el color de los links
\usepackage{hyperref}
\hypersetup{hidelinks}

% Generamos un comando para saltar pagina con las secciones
\NewDocumentCommand{\cpsection}{s o m}{%
  \clearpage
  \IfBooleanTF{#1}
    {\section*{#3}}
    {%
      \IfNoValueTF{#2}
        {\section{#3}}
        {\section[#2]{#3}}%
    }%
}

\NewDocumentCommand{\cpsubsection}{s o m}{%
  \clearpage
  \IfBooleanTF{#1}
    {\subsection*{#3}}
    {%
      \IfNoValueTF{#2}
        {\subsection{#3}}
        {\subsection[#2]{#3}}%
    }%
}

% Elimina la palabra "Capítulo" de los títulos de los capítulos
\titleformat{\chapter}[display]
  {\normalfont\bfseries}{}{0pt}{\Huge\thechapter.\space}

\titleformat{name=\chapter,numberless}[display]
  {\normalfont\bfseries}{}{0pt}{\Huge}

\titlespacing*{\chapter}{0pt}{-50pt}{20pt}

% Personalización del índice de listados
\renewcommand{\lstlistingname}{Código}  % Cambiar el nombre de "Listing" a "Código"
\renewcommand{\lstlistlistingname}{Índice de códigos}

% Añade numeración a los subsubsections y los añade al índice
\setcounter{secnumdepth}{4}
\setcounter{tocdepth}{4}

% Idioma predeterminado (Español)
\selectlanguage{spanish}

\begin{document}
  \begin{titlepage}
      \centering
      \includegraphics[width=0.6\textwidth]{./.img/logo.jpg}\\
      \vspace{1cm}
      \Large Ingeniería Informática de Gestión y Sistemas de Información\\
      \vspace{3cm}
      \Huge Desarrollo Avanzado de Software\\
      \vspace{0.5cm}
      \huge \textbf{LibreBook}\\
      \vspace{7.5cm}
      \Large Estudiante:\\
      \vspace{0.2cm}
      \large Gabiña Barañano, Xabier\\
      \vspace{1cm}
      \vfill
      \today
  \end{titlepage}
  \tableofcontents
  \listoffigures
  \listoftables
  \lstlistoflistings
  \chapter{Introducción}
  \chapter{Objetivos}
    \section{Elementos obligatorios}
    \begin{longtable}{|c|p{7.5cm}|p{7.5cm}|}
      \hline
      \textbf{ID} & \textbf{Descripción} & \textbf{Funcionalidad}\\
      \hline
      \endhead
      1 & Uso de ListView+CardView personalizado o de RecyclerView+CardView para mostrar listados de elementos con diferentes características. & \\
      \hline
      2 & Usar una base de datos local, para listar, añadir y modificar elementos y características de cada elemento. & \\
      \hline
      3 & Uso de diálogos. & \\
      \hline
      4 & Usar notificaciones locales. & \\
      \hline
      5 & Control de la pila de actividades. & \\
      \hline
    \end{longtable}
    \section{Elementos opcionales}
  \chapter{Planificación}
  \chapter{Manual de usuario}
    \section{Registrarse}
    \section{Iniciar sesión}
    \section{Buscar libros}
    \section{Añadir libros a tu biblioteca}
    \section{Ver tu biblioteca}
    \section{Ajustes}
  \chapter{Dificultades}
  \chapter{Conclusiones}
\end{document}